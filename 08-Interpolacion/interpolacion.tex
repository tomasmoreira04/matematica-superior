\documentclass[11pt]{article}

\usepackage[margin=1in, headheight=14.5pt]{geometry}
\usepackage{amsfonts, amsmath, amssymb}
\usepackage{enumitem}
\usepackage[none]{hyphenat}
\usepackage{fancyhdr}
\usepackage[spanish, es-noshorthands]{babel}
\usepackage[spanish, calc]{datetime2}
\usepackage{fmtcount}
\usepackage{graphicx}
\usepackage{float}
\usepackage[nottoc, notlot, notlof]{tocbibind}
\usepackage{tocloft}
\usepackage[utf8]{inputenc}
\usepackage{parskip}
\usepackage{xcolor}
\usepackage{cancel}
\usepackage{textcomp}
\usepackage{pgfplots}
\usepackage{tikz}
\usetikzlibrary{shapes.misc}
\usepackage{polynom}
\usetikzlibrary{datavisualization}
\usetikzlibrary{datavisualization.formats.functions}
\pgfplotsset{compat=1.15}
\usepackage{mathrsfs}
\usetikzlibrary{arrows}

\newcommand{\dropsign}[1]{\smash{\llap{\raisebox{-.5\normalbaselineskip}{$#1$\hspace{2\arraycolsep}}}}}%


\parindent 0ex

\pgfplotsset{width=10cm,compat=1.9}

\def\imj{\mathrm{j}}
\def\sen{\mathrm{sen}}

\newcommand{\lapl}[1]{\mathscr{L} \left\lbrace {#1} \right\rbrace}
\newcommand{\ilapl}[1]{\mathscr{L}^{-1} \left\lbrace {#1} \right\rbrace}

\renewcommand\cftsecleader{\cftdotfill{\cftdotsep}}
\renewcommand{\baselinestretch}{1.1}
\newcommand*\circled[1]{\tikz[baseline=(char.base)]{
		\node[shape=circle,draw,inner sep=2pt] (char) {#1};}}
	
\newcommand{\highlight}[2]{\colorbox{#1}{$\displaystyle #2$}}

\graphicspath{{\ProjectRoot/commons/img/}}

\newcommand*{\ProjectRoot}{../../matematica-superior}


\begin{document}
		
	\begin{titlepage}
		\begin{center}
			\vspace*{0.5cm}
			\Large{\textbf{Universidad Tecnológica Nacional}}\\
			\Large{\textbf{Facultad Regional Buenos Aires}}\\
			\begin{center}
				\includegraphics[scale=0.4]{logoutn.png}
			\end{center}
			\vfill
			\line(1,0){400}\\
			\vspace*{0.3cm}
			\huge{\textbf{Matemática Superior}}\\
			\Large{\textbf{Unidad 7: Sistemas de ecuaciones lineales}}\\
			\large{Ejercicios resueltos}
			\line(1,0){400}\\
			\vfill
			Tomás Moreira \\
			
			\DTMnewdatestyle{mydate}{%
				\renewcommand{\DTMdisplaydate}[4]{%
					\DTMMonthname{##2} \number##1
				}
				\renewcommand{\DTMDisplaydate}{\DTMdisplaydate}
			}
			
			\DTMsetdatestyle{mydate}
			\today
				
				
		\end{center}
	\end{titlepage}

	\tableofcontents
	\thispagestyle{empty}
	\clearpage

	\setcounter{page}{1}
	
	\section{Ejercicio 1 (tipos de matrices)}
	Dadas las siguientes matrices: $\;\;\;$
	$A=\begin{bmatrix}
		-2 & 1 \\
		1 & -3
	\end{bmatrix} \;\;\;
	B=\begin{bmatrix}
		7 & 2 & 0 \\
		3 & 5 & -1 \\
		0 & 5 & -6 \\
	\end{bmatrix} \;\;\;
	C=\begin{bmatrix}
		2 & -1 & 0 \\
		-1 & 4 & 2 \\
		0 & 2 & 2 \\
	\end{bmatrix}
	$
	
	Analice si cada una de ellas es:
	\begin{enumerate}[label=\alph*)]
		\item Simétrica
		\item Singular
		\item Dominante diagonalmente (especifique si lo es en sentido estricto)
	\end{enumerate}

	Arrancamos con esta unidad, primero recorreremos ejercicios que son más "básicos" de matrices, y luego nos metemos de lleno con los métodos iterativos.
	
	Comencemos para la matriz $A$:
	
	\textbf{Para el inciso a}, de Álgebra, recordemos que una matriz es simétrica si $A=A^t$.\\
	Trasponemos la matriz, quedando: $A^t=\begin{bmatrix}
		-2 & 1 \\
		1 & -3
	\end{bmatrix}$, como $A=A^t$, la matriz \fcolorbox{black}{yellow}{es simétrica}.

	\textbf{Para el inciso b}: una matriz es singular si su determinante es igual a 0.
	
	$det(A)=\begin{vmatrix}
		-2 & 1 \\
		1 & -3
	\end{vmatrix}=(-2)\cdot(-3)-1\cdot1=5\ne0$, por ende, \fcolorbox{black}{yellow}{no es singular}.

	\textbf{Para el inciso c}, hay que recordar la definición de matriz diagonal dominante:\\
	Una matriz es diagonal dominante si: $\displaystyle |a_{ii}|\ge \sum_{j=1, i\ne j}^{n} |a_{ij}| \; \forall i=1\dots n$
	
	Y la matriz será estricamente diagonal dominante si $\displaystyle |a_{ii}| > \sum_{j=1, i\ne j}^{n} |a_{ij}| \; \forall i=1\dots n$, es decir no se considera la igualdad, sino que es mayor estricto.
	
	Dicho en criollo, nos está queriendo decir que el valor absoluto de un elemento de la diagonal principal debe ser mayor o igual a la suma de los valores absolutos de los elementos de la misma fila, y esto debe ocurrir en todas las filas.
	
	Para nuestro caso: $\begin{bmatrix}
		-2 & 1 \\
		1 & -3
	\end{bmatrix}$$\begin{matrix}
	\rightarrow |-2| \ge |1| \\
	\rightarrow |-3| \ge |1|
\end{matrix}$

	Como resultan mayores todos los elementos de la diagonal, \fcolorbox{black}{yellow}{es diagonalmente dominante} y, además, lo es en \fcolorbox{black}{yellow}{forma estricta}.\\
	
	Ahora, comencemos con la matriz $B=\begin{bmatrix}
		7 & 2 & 0 \\
		3 & 5 & -1 \\
		0 & 5 & -6 \\
	\end{bmatrix}$.
	
	Inciso a:
	
	$B^t=\begin{bmatrix}
		7 & 3 & 0 \\
		2 & 5 & 5 \\
		0 & -1 & -6 \\
	\end{bmatrix} \;\;\;$, como $B\ne B^t$ la matriz \fcolorbox{black}{yellow}{no es simétrica}.

	Inciso b:
	
	$det(B)=\begin{vmatrix}
		7 & 2 & 0 \\
		3 & 5 & -1 \\
		0 & 5 & -6 \\
	\end{vmatrix} = [7 \cdot 5 \cdot (-6) + 2 \cdot (-1) \cdot 0 + 3 \cdot 5 \cdot 0] - [0 \cdot 5 \cdot 0 + 5 \cdot (-1) \cdot 7 + 3 \cdot 2 \cdot (-6)]$\\
	$= -210 + 71 = -139 \ne 0 \rightarrow $ no es singular.
	
	Inciso c:$
	\begin{bmatrix}
		7 & 2 & 0 \\
		3 & 5 & -1 \\
		0 & 5 & -6 \\
	\end{bmatrix}$
	$\begin{matrix}
	\rightarrow |7| \ge |2| + |0|\\
	\rightarrow |5| \ge |3| + |-1| \\
	\rightarrow |-6| \ge |5| + |0| \\
	\end{matrix}$

	Las desigualdades son todas ciertas, por ende, la matriz \fcolorbox{black}{yellow}{es diagonal dominante}, y lo es en \fcolorbox{black}{yellow}{forma estricta}.\\
	
	Por último, veamos la matriz $C=\begin{bmatrix}
		2 & -1 & 0 \\
		-1 & 4 & 2 \\
		0 & 2 & 2 \\
	\end{bmatrix}$

	Inciso a:
	
	$C^t=\begin{bmatrix}
		2 & -1 & 0 \\
		-1 & 4 & 2 \\
		0 & 2 & 2 \\
	\end{bmatrix}$, como $C=C^t$, la matriz es simétrica.

	Inciso b: $|C|=16-10=6$, por ende no es singular,
	
	Inciso c:
	
	$\begin{bmatrix}
		2 & -1 & 0 \\
		-1 & 4 & 2 \\
		0 & 2 & 2 \\
		\end{bmatrix}$$
	\begin{matrix}
		\rightarrow |2| \ge |-1| + |0| \\
		\rightarrow |4| \ge |-1| + |2| \\
		\rightarrow |2| \ge |0| + |2| \\
		\end{matrix}
	$
	
	Como vemos la matriz es diagonal dominante, aunque en este caso no lo es en sentido estricto, debido a que en la tercer fila la suma es igual al valor absoluto del elemento de la diagonal principal.
	
	\section{Ejercicio 2 (radio espectral)}
	Dadas las siguientes matrices: $A=\begin{bmatrix}
		2 & -1\\
		-1 & 2
	\end{bmatrix} \;\;\; B=\begin{bmatrix}
	2 & 1 & 0 \\
	1 & 2 & 0 \\
	0 & 0 & -5 \\
\end{bmatrix}$

	Calcule:
	\begin{enumerate}[label=\alph*)]
		\item Los autovalores
		\item El radio espectral
		\item Las normas $\lVert A \rVert_{2}$ y $\lVert B \rVert_{2}$
	\end{enumerate}

	Empecemos con la matríz $A=\begin{bmatrix}
		2 & -1\\
		-1 & 2
	\end{bmatrix}$

	\textbf{Para el inciso a}, debemos hacer $|A-\lambda I|=0$, es decir:
	
	$|A-\lambda I|=\begin{vmatrix}
		2 - \lambda & -1\\
		-1 & 2 \lambda
	\end{vmatrix} = (2-\lambda)(2-\lambda) - [(-1)(-1)]=4-2\lambda-2\lambda+\lambda^2-1=\boxed{\lambda^2-4\lambda+3}$
	
	Al resolver las raíces del polinomio característico, obtenemos los autovalores: \fcolorbox{black}{yellow}{$\lambda_1=3\;,\lambda_2=1$}
	
	\textbf{Para el inciso b}, por definición el radio espectral es $\rho(A)=|\text{máx} \lambda_i|$. En este caso es $\rho(A)=3$.
	
	Por último, \textbf{para el inciso c}, debemos recordar que la norma 2 de una matriz es: $\sqrt{\rho(A^t\cdot A)}$, entonces debemos calcular $A^t \cdot A$
	
	$A^t\cdot A = \begin{pmatrix}
		2 & -1\\
		-1 & 2
	\end{pmatrix} \cdot \begin{pmatrix}
	2 & -1\\
	-1 & 2
\end{pmatrix}=\begin{pmatrix}
5 & -4\\
-4 & 5
\end{pmatrix}$

	Ahora, habría que calcular el radio de espectral a esta última matriz:
	$|(A^tA)-\lambda I|=\begin{vmatrix}
		5 - \lambda & -4\\
		-4 & 5-\lambda
	\end{vmatrix}=(5-\lambda)(5-\lambda)-(-4)(-4)=\lambda^2-10\lambda+9=0$, las soluciones son: $\lambda_1=9 \;, \lambda_2=1$

	El radio espectral es, en efecto, 9.
	
	Por último, $\lVert A \rVert_2=\sqrt{\rho(A^tA)}=\sqrt{9}=3$.
	
	-- mostrar el cálculo de autovalores con wolfram
	
	\section{Ejercicio 3 (normas)}
	Dada la matriz $A=\begin{bmatrix}
		5 & -5 & -7\\
		-4 & 2 & -4 \\
		-7 & -4 & 5
	\end{bmatrix}$. Halle las normas $\lVert A \rVert_1$ y $\lVert A \rVert_\infty$

	Este ejercicio es más sencillo que los anteriores, que requieren más cálculos.
	
	Tenemos que sumar las filas (en valor absoluto), para el caso de la norma infinito, y análogo pero con las columnas para la norma 1.
	
	$\begin{bmatrix}
		5 & -5 & -7\\
		-4 & 2 & -4 \\
		-7 & -4 & 5
	\end{bmatrix}$$\begin{matrix}
	\rightarrow |5| + |-5| + |-7| = 17 \\
	\rightarrow |-4| + |2| + |-4| = 10 \\
	\rightarrow |-7| + |-4| + 5 = 16
\end{matrix}$\\$
\begin{matrix}
	\;\;\;\downarrow & \downarrow & \downarrow\\
	\;\;\;16 & 11 & 16\\
\end{matrix}$
		
		Luego de hacer las sumas, el máximo de esos valores será la norma correspondiente. Para el caso de la norma infinito es $\lVert A \rVert_\infty=17$, y para la norma 1 es $\lVert A \rVert_1=16$.
		
		Una regla mnemotécnica es mirar la forma de los símbolos, el infinito es alargado, por ende, sumo los elementos de toda una fila; mientras que el uno está a lo alto, entonces sumo las columnas.
		
	\section{Ejercicio 4 (diagonal dominante)}
	Dadas las matrices $A=\begin{pmatrix}
	2k-3 & 4 & 1 \\
	2 & 5 & 2 \\
	2 & 4 & 10-k	
	\end{pmatrix} \;\;$ y $B=\begin{pmatrix}
	3k-3 & 4 & -1 \\
	2 & 5+k^2 & -2 \\
	7 & 4 & 10-k
	\end{pmatrix}$
	
	\begin{enumerate}[label=\alph*)]
		\item Indique los valores de $k \in \mathbb{R}$ tal que sean diagonalmente dominantes.
		\item ¿Para qué valores de $k \in \mathbb{R}$, son estricamente diagonal dominantes?
	\end{enumerate}

	Los incisos a y b son muy parecidos, solo va a variar un signo de los que vamos a usar. Empecemos para la matriz A:
	
	Sabemos que para ser diagonal dominante deben ser los módulos de los elementos de la diagonal principal mayores a la suma de los módulos de los valores de la misma fila, es decir, debemos plantear:
	
	$|2k-3| \ge 4 + 1 \;\;$ y $\;\;|10-k| \ge 2 + 4$
	
	$|2k-3| \ge 5$ 
	
	$2k-3 \ge 5 \;\; \lor \;\; 2k-3\le -5 $
	
	$\boxed{k \ge 4 \;\; \lor \;\; k \le -1}$
	
	$|10-k| \ge 6$
	
	$10-k \ge 6 \;\; \lor \;\; 10-k\le -6 $
	
	$\boxed{k \le 4 \;\; \lor \;\; k \ge 16}$
	
	Tenemos que hacer la intersección de estos dos conjuntos. Termina quedándonos:\\ \fcolorbox{black}{yellow}{$k \in (-\infty; -1] \cup \{4\} \cup [16; +\infty)$}
	
	Luego, para el \textbf{inciso b} lo que cambia es el signo de la desigualdad, que ya no cuenta con el igual, de modo que quedan:
	
	$\boxed{k > 4 \;\; \lor \;\; k < -1}$
	
	$\boxed{k < 4 \;\; \lor \;\; k > 16}$
	
	Al hacer la intersección, el 4 deja de ser solución y no incluimos los extremos como solución:\\
	\fcolorbox{black}{yellow}{$k \in (-\infty; -1) \cup (16; +\infty)$}\\
	
	Vamos con la matriz $B=\begin{pmatrix}
		3k-3 & 4 & -1 \\
		2 & 5+k^2 & -2 \\
		7 & 4 & 10-k
	\end{pmatrix}$

	Planteamos lo mismo que el anterior:
	
	$|3k-3| \ge 5$
	
	$3k-3 \ge 5 \;\; \lor \;\; 3k-3 \le -5$
	
	$\boxed{k \ge \frac{8}{3} \;\; \lor \;\; k \le -\frac{2}{3}}$
	
	$\boxed{|5+k^2|\ge 4 \rightarrow$ válida $\forall k}$
	
	$|10-k|\ge 11$
	
	$10-k \ge 11 \;\; \lor \;\; 10-k \le -11$
	
	$\boxed{k \le -1 \;\; \lor \;\; k\ge21}$
	
	Al hacer la intersección de los dos conjuntos: \fcolorbox{black}{yellow}{$k \in (-\infty; -1] \cup [21; +\infty)$}.
	
	Por último, para la estricta, solo hay que eliminar los extremos de la solución, quedando: \\ \fcolorbox{black}{yellow}{$k \in (-\infty; -1) \cup (21; +\infty)$}
	
	\section{Ejercicio 5 (método de Jacobi)}
	Aplique el método iterativo de Jacobi para resolver el sistema:
	$$\begin{cases}
		7x_1-x_2+4x_3=8 \\
		3x_1-8x_2+2x_3=-4 \\ 
		4x_1+x_2-6x_3=3
	\end{cases}$$

	realizando los cálculos con tres decimales redondeados e iterando hasta que cumpla que:
	$$\left\lVert x_{i}^{(r+1)}-x_{i}^{(r)} \right\rVert_\infty < 0.03$$
	
	Para los métodos iterativos lo que hay que hacer es despejar una variable en cada una de las ecuaciones. Es decir, que en la primera ecuación debemos despejar $x_1$, en la segunda $x_2$ y en la última $x_3$, de esta manera:
	
	$\displaystyle \begin{cases}
		x_1^{(r+1)}=\hspace{0.8cm}+\frac{1}{7}x_2^{(r)}-\frac{4}{7}x_3^{(r)}+\frac{8}{7}\\
		x_2^{(r+1)}=\frac{3}{8}x_1^{(r)} \hspace{1.1cm}+\frac{1}{4}x_3^{(r)}+\frac{1}{2}\\
		x_3^{(r+1)}=\frac{2}{3}x_1^{(r)}+\frac{1}{6}x_2^{(r)} \hspace{1.1cm}-\frac{1}{2}
	\end{cases}$

	Teniendo estas ecuaciones, tenemos que ahora, usar un vector inicial e ir usándolo para calcular los valores. Por lo general, el vector inicial es el vector nulo, en este caso sería $x^{(0)}=(0;0;0)$
	
	Con esta info, vamos al excel.

	Alerta spoiler: el método converge en 8 iteraciones con el error deseado
	
	\section{Ejercicio 6 (método de Gauss-Seidel + matrices T)}
	Resuelva el sistema anterior usando el método de Gauss-Seidel. Halle las $T_J$ y $T_{GS}$
	
	En efecto, hay que proceder de la misma manera que en el ejercicio anterior, hay que despejar las variables en las ecuaciones correspondientes. Lo que cambian son las fórmulas iterativas, a decir:
	
	$\displaystyle \begin{cases}
		x_1^{(r+1)}=\hspace{1.3cm}+\frac{1}{7}x_2^{(r)}-\frac{4}{7}x_3^{(r)}+\frac{8}{7}\\
		x_2^{(r+1)}=\frac{3}{8}x_1^{(r+1)} \hspace{1.2cm}+\frac{1}{4}x_3^{(r)}+\frac{1}{2}\\
		x_3^{(r+1)}=\frac{2}{3}x_1^{(r+1)}+\frac{1}{6}x_2^{(r+1)} \hspace{0.8cm}-\frac{1}{2}
	\end{cases}$

	Como vemos, tanto para $x_2$ como para $x_3$ lo que pasa es que se usan valores que ya estamos calculando, es decir, estamos acelerando el método.
	
	Veamos como impacta en la rapidez de la convergencia en el excel.
	
	-----
	
	Vemos que converge en la mitad de iteraciones. Y esto es una tendencia que no es casualidad, \textbf{por lo general, el método de Gauss-Seidel es un 50\% más rápido que el método de Jacobi}.
	
	Para el punto de hallar las matrices $T_J$ y $T_{GS}$, debemos recordar:
	
	Para un sistema definido como $A \bullet X = B$, con $A$ siendo la matriz de coeficientes y $B$ la matriz de términos independientes. Se puede descomponer a $A$ como $A=D-L-U$
	
	$T_J=D^{-1}\cdot (L+U)$, donde $D$ es la matriz diagonal de la matriz de coeficientes, y las matrices $L$ y $U$ son las matrices triangulares inferiores y superiores asociadas a esa misma matriz, cambiadas de signo, respectivamente.
	
	$T_{GS}=(D-L)^{-1}U$
	
	Estas matrices sirven mucho para calcular de forma iterativa los métodos si uno quiere programarlas en algún lenguaje, o bien hacerlo en un Excel. Recuerden que las fórmulas iterativas son:
	
	$x^{(k+1)}=T_J\cdot x^{(k)} + C_J$, con $C_J=(D^{-1}\cdot B)$, para el método de Jacobi
	
	$x^{(k+1)}=T_{GS} \cdot x^{(k)} + C_{GS}$, con $C_{GS}=(D-L)^{-1}\cdot B$, para el método de Gauss-Seidel
	
	\section{Ejercicio 7 (Gauss Seidel + DD)}
	Sea el S.E.L.: $\begin{cases}
		4x-y=1\\
		x+3y=2
	\end{cases}$
	\begin{enumerate}[label=\alph*)]
		\item Indique si los métodos iterativos convergen. Justifique.
		\item Realice cinco iteraciones por el método de Gauss-Seidel con $F(4, 10, 34, 56)$ y redondeo simétrico partiendo de $X^{(0)}=(0;0)$.
		\item Idem b) pero partiendo de $X^{(0)}=(9;5)$. Compare.
	\end{enumerate}

	En este ejercicio vamos a abordar un tema que hasta ahora en la práctica habíamos ignorado. ¿Los métodos de Gauss-Seidel y Jacobi..siempre convergen? La respuesta es que no. En la teoría se ven 3 criterios.
	
	\begin{itemize}
		\item Una sucesión definida por $X^{(k)}=TX^{(k-1)} + C$, con $k\ge 1$ y $C\ne0$ converge a la solución única de este sistema SI Y SOLO SI (\textbf{condición necesaria y suficiente}) $\rho(T)<1$. También, convergen con cualquier vector inicial
		\item Si $\lVert T \rVert<1$ entonces los métodos iterativos convergen (esto es una condición suficiente). Converge con cualquier vector inicial.
	\end{itemize}

	Sin embargo, estas condiciones pueden ser un tanto incómodas para calcular, por lo que este teorema es el que vamos a utilizar, generalmente:
	
	\textbf{Si $A$ es una matriz diagonalmente dominante para cualquier $X_0$ los métodos de Jacobi o Gauss Seidel dan sucesión ${X^{(k)}}_{k=0}^{\infty}$ que converge a la solución de $AX=B$}
	
	Vale la pena destacar que la anterior es una \textbf{\underline{condición suficiente}}.
	
	Con esto dicho, vamos a verificar que la matriz A de coeficientes del ejercicio sea diagonalmente dominante.
	
	$A=\begin{pmatrix}
		4 & -1 \\
		1 &3
	\end{pmatrix}$$\begin{matrix}
	\rightarrow 4 > 1 \\
	\rightarrow 3 > 1
\end{matrix}$

	Vemos que la matriz es diagonal dominante, por lo tanto podemos asegurar que los métodos iterativos convergen \textbf{(inciso a)}.
	
	\textbf{Para el inciso b} debemos realizar los despejes:
	
	$\begin{cases}
		x^{(k+1)}=\frac{1}{4}y^{(k)}+\frac{1}{4}\\
		y^{(k+1)}=\frac{-1}{3}x^{(k+1)}+\frac{2}{3}
	\end{cases}$

	Con estos despejes vamos al Excel a hacer las 6 iteraciones.
	
	En cuanto al inciso c, la comparación que podemos hacer que ambos convergen, solo que el del ejercicio c tarda una iteración más.
	
	\section{Ejercicio 9c (intercambio + Gauss Seidel)}
	Aplique el método iterativo de Gauss Seidel para encontrar una solución aproximada de los siguientes sistemas, iterando hasta que cumpla con $\left\lVert x_i^{(r+1)}-x_i^{(r)} \right \rVert_\infty < 10^{-3}$
	
	Por abuso de notación, reemplazo $x_1$ por $x$, $x_2$ por $y$ y $x_3$ por $z$.
	
	$\begin{cases}
		x+6y+2z=15\\
		x+y-6z=-3\\
		6x+y+z=9
	\end{cases}$

	Si intentamos hacer la verificación de si la matriz de coeficientes es diagonal dominante nos vamos a encontrar con que no lo es. Ya en la primera fila nos encontramos que $1\ngeq6+2$.
	
	A no desesperarse! Si la matriz no es DD, podemos intentar que lo sea! ¿Cómo? ...\\
	...\\
	...\\
	Intercambiando filas!!! (o columnas, aunque esto es menos común).
	
	Hagamos el siguiente intercambio: la primera pasa a ser la segunda, la segunda pasa a ser la tercera, y la última pasa a ser la primera.
	
	$
	\begin{cases}
		6x+y+z=9\\
		x+6y+2z=15\\
		x+y-6z=-3
	\end{cases}
	$
	
	Ahora la matríz de coeficientes es diagonalmente dominante! Ya que $\boxed{6>1+1}$, $\boxed{6>2+1}$ y $\boxed{6>1+1}$
	
	Realizamos el despeje y resolvemos:
	
	$\begin{cases}
		x^{(k+1)}=\frac{9}{6}-\frac{1}{6}y^{(k)}-\frac{1}{6}z^{(k)}\\
		y^{(k+1)}=\frac{15}{6}-\frac{1}{6}x^{(k+1)}-\frac{2}{6}z^{(k)}\\
		z^{(k+1)}=\frac{3}{6}+\frac{1}{6}x^{(k+1)}+\frac{1}{6}x^{(k+1)}
	\end{cases}$

	\section{Ejercicio 12 (no DD)}
	Sea el sistema $A\bullet X = B$ con $A=\begin{pmatrix}
		5 & 7\\
		13 & 20
	\end{pmatrix}\;\;\;$ $B=\begin{pmatrix}
	29\\
	79
\end{pmatrix}$

	\begin{enumerate}[label=\alph*)]
		\item Indique si la matriz es diagonal dominante. ¿Puede asegurarse que sea posible resolverlo por el método de Gauss Seidel?
		\item Realice 100 iteraciones por Gauss Seidel partiendo de $X^{0}=\begin{pmatrix}
			1\\4
		\end{pmatrix}$. ¿Converge a la solución que es $(3;2)$?
		\item Analice si el resultado anterior contradice al teorema.
	\end{enumerate}

	Bien, para el inciso a, podemos ver a simple vista que la matriz no es diagonal dominante. Y no solo eso, no hay manera de reordenarlo para que lo sea.
	
	Vamos a ver los resultados de las 100 iteraciones en el Excel.
	
	Vemos que de todas maneras converge, a pesar de que la matriz no es diagonal dominante. Esto ocurre (damos respuesta al inciso c) porque el teorema es condición suficiente, no necesaria; por ende aunque no sea DD puede igual converger.
\end{document}
 