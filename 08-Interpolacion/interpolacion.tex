\documentclass[11pt]{article}

\usepackage[margin=1in, headheight=14.5pt]{geometry}
\usepackage{amsfonts, amsmath, amssymb}
\usepackage{enumitem}
\usepackage[none]{hyphenat}
\usepackage{fancyhdr}
\usepackage[spanish, es-noshorthands]{babel}
\usepackage[spanish, calc]{datetime2}
\usepackage{fmtcount}
\usepackage{graphicx}
\usepackage{float}
\usepackage[nottoc, notlot, notlof]{tocbibind}
\usepackage{tocloft}
\usepackage[utf8]{inputenc}
\usepackage{parskip}
\usepackage{xcolor}
\usepackage{cancel}
\usepackage{textcomp}
\usepackage{pgfplots}
\usepackage{tikz}
\usetikzlibrary{shapes.misc}
\usepackage{polynom}
\usetikzlibrary{datavisualization}
\usetikzlibrary{datavisualization.formats.functions}
\pgfplotsset{compat=1.15}
\usepackage{mathrsfs}
\usetikzlibrary{arrows}
\usepackage{adjustbox}

\newcommand{\dropsign}[1]{\smash{\llap{\raisebox{-.5\normalbaselineskip}{$#1$\hspace{2\arraycolsep}}}}}%


\parindent 0ex

\pgfplotsset{width=10cm,compat=1.9}

\def\imj{\mathrm{j}}
\def\sen{\mathrm{sen}}

\newcommand{\lapl}[1]{\mathscr{L} \left\lbrace {#1} \right\rbrace}
\newcommand{\ilapl}[1]{\mathscr{L}^{-1} \left\lbrace {#1} \right\rbrace}

\renewcommand\cftsecleader{\cftdotfill{\cftdotsep}}
\renewcommand{\baselinestretch}{1.1}
\newcommand*\circled[1]{\tikz[baseline=(char.base)]{
		\node[shape=circle,draw,inner sep=2pt] (char) {#1};}}
	
\newcommand{\highlight}[2]{\colorbox{#1}{$\displaystyle #2$}}

\graphicspath{{\ProjectRoot/commons/img/}}

\newcommand*{\ProjectRoot}{../../matematica-superior}


\begin{document}
		
	\begin{titlepage}
		\begin{center}
			\vspace*{0.5cm}
			\Large{\textbf{Universidad Tecnológica Nacional}}\\
			\Large{\textbf{Facultad Regional Buenos Aires}}\\
			\begin{center}
				\includegraphics[scale=0.4]{logoutn.png}
			\end{center}
			\vfill
			\line(1,0){400}\\
			\vspace*{0.3cm}
			\huge{\textbf{Matemática Superior}}\\
			\Large{\textbf{Unidad 8: Interpolación}}\\
			\large{Ejercicios resueltos}
			\line(1,0){400}\\
			\vfill
			Tomás Moreira \\
			
			\DTMnewdatestyle{mydate}{%
				\renewcommand{\DTMdisplaydate}[4]{%
					\DTMMonthname{##2} \number##1
				}
				\renewcommand{\DTMDisplaydate}{\DTMdisplaydate}
			}
			
			\DTMsetdatestyle{mydate}
			\today
				
				
		\end{center}
	\end{titlepage}

	\tableofcontents
	\thispagestyle{empty}
	\clearpage

	\setcounter{page}{1}
	
	\section{Ejercicio 3 (Lagrange)}
	Sean los siguientes puntos: $(0,0), (1/6, 1/2) \text{ y } (1/2, 1)$
	
	\begin{enumerate}[label=\alph*)]
		\item Verifique que dichos puntos pertenecen a la función $f(x)=\sen(\pi x)$
		\item Halle el polinomio interpolante de Lagrange.
		\item Calcule $\sen(\pi/3)$ mediante el polinomio hallado en b)
		\item Estime el error que se comete en c)
	\end{enumerate}

	\textbf{Inciso a):}\\
	$f(x)=\sen(\pi x)$
	
	$f(0)=\sen(0)=0$\\
	$f(1/6)=\sen(1/6)=1/2$\\
	$f(1/2)=\sen(1/2)=1$
	
	Todos los puntos verifican
	
	\textbf{Inciso b):}
	
	Para este punto hay que recordar la fórmula del polinomio interpolante de Lagrange:
	$$\displaystyle P(x)=\sum_{i=0}^{n}\frac{y_i}{L_i(x_i)}L_i(x)$$
	Los $L_i(x)$ son los factores $(x-x_k)$ pero sin incluir al $x_i$.
	
	De este modo, para nuestro caso los $L$ son:
	
	$L_0(x)=(x-1/6)(x-1/2) \rightarrow L_0(x_0)=L(0)=1/12$\\
	$L_1(x)=(x-0)(x-1/2) \rightarrow L_1(x_1)=L(1/6)=-1/18$\\
	$L_2(x)=(x-0)(x-1/6) \rightarrow L_2(x_2)=L(1/2)=1/6$
	
	Entonces nuestro polinomio de Lagrange será:
	
	$\displaystyle P(x)=\frac{0}{1/12}(x-1/6)(x-1/2)-\frac{1/2}{1/18}x(x-1/2)+\frac{1}{1/6}x(x-1/6)=-9x(x-1/2)+6x(x-1/6)$
	
	\textbf{Inciso c):}
	
	$\displaystyle P(1/3)=-9\cdot\frac{1}{3}\left( \frac{1}{3}-\frac{1}{2} \right) + 6 \cdot \frac{1}{3}\left(\frac{1}{3}-\frac{1}{6}\right)=\frac{5}{6}=0.8333333$
	
	\textbf{Inciso d):}
	
	Para estimar el error, debemos ver el valor real:
	
	$sen(\pi/3)=0.866025...$
	
	El error es entonces: $e=0.866025-0.833333=0.032692$
	
	\section{Ejercicio 4 (Lagrange)}
	Sean $\alpha, \beta, \gamma, \delta \in \mathbb{R}$. ¿Qué relación debe cumplirse entre $\alpha, \beta, \gamma \text{ y } \delta$ para que exista un polinomio real $p(x)$ de grado no mayor a 2 tal que $p(-1)=\alpha$, $p(1)=\beta$, $p(3)=\gamma$ y $p(0)=\delta$?
	
	Vamos a plantear el polinomio de Lagrange para los datos del ejercicio.
	
	Primero, tabulemos:
	
	\begin{tabular}{|c|c|c|c|c|}
		\hline
		i & 0 & 1 & 2 & 3\\
		\hline
		$x_i$ & -1 & 0 & 1 & 3\\
		\hline
		$y_i$ & $\alpha$ & $\delta$ & $\beta$ & $\gamma$\\
		\hline
	\end{tabular}

	Planteamos:
	
	$L_0(x)=x(x-1)(x-3) \rightarrow L_0(x_0)=L_0(-1)=-8$\\
	$L_1(x)=(x+1)(x-1)(x-3) \rightarrow L_1(x_1)=L_1(0)=3$\\
	$L_2(x)=(x+1)x(x-3) \rightarrow L_2(x_2)=L_2(1)=-4$\\
	$L_3(x)=(x+1)x(x-1) \rightarrow L_3(x_3)=L_3(3)=24$
	
	Queda el polinomio:
	
	$\displaystyle P(x)=\frac{\alpha}{-8}x(x-1)(x-3)+\frac{\delta}{3}(x+1)(x-1)(x-3)+\frac{\beta}{-4}(x+1)x(x-3)+\frac{\gamma}{24}(x+1)x(x-1)$
	
	Ahora, como nos piden un polinomio de grado no mayor a 2 (es decir, menor o igual a 2), nos interesa que se anule el coeficiente cúbico. Para eso, debe cumplirse:
	
	$\displaystyle \frac{\alpha}{-8}+\frac{\delta}{3}+\frac{\beta}{-4}+\frac{\gamma}{24}=0$
	
	Si multiplicamos por 24, tenemos la respuesta: $-3\alpha+8\delta-6\beta+\gamma=0$
	
	\section{Ejercicio 7 (Newton-Gregory)}
	En la siguiente tabla se listan los valores de $f(x)=\sqrt x$ redondeados hasta 5 decimales.
	
	\begin{tabular}{|c|c|c|c|c|c|c|c|}
		\hline
		i & 0 & 1 & 2 & 3 & 4 & 5 & 6 \\
		\hline
		$x_i$ & 1.00 & 1.05 & 1.10 & 1.15 & 1.20 & 1.25 & 1.30\\
		\hline
		$y_i$ & 1.0000 & 1.02470 & 1.04881 & 1.07238 & 1.09544 & 1.11803 & 1.14017\\
		\hline
	\end{tabular}

\begin{enumerate}[label=\alph*)]
	\item Calcule las diferencias hasta orden sexto
	\item Interpole las raíces de $1.01$ y $1.28$ usando las fórmulas de Newton.
	\item Aplique la fórmula de Lagrange para obtener la raíz cuadrada de $1.12$. \textit{(no lo vamos a hacer)}
	\item Teniendo en cuenta que $\sqrt{x}=1.05$, utilice los datos de la tabla para encontrar $x$.
	\item Interpolar el valor $\sqrt{1.32}$.
\end{enumerate}

	\textbf{Inciso a):}\\
	\begin{tabular}{|c c c c c c c c c|}
		\hline
		$i$ & $x_i$ & $y_i$ & O1 & O2 & O3 & O4 & O5 & O6\\
		\hline
		0 & 1.00 & 1.0000  &        &        &         &          & &\\
		  &      &         & 0.494  &        &         &          & &\\
		1 & 1.05 & 1.02470 &        & -0.118 &         &          & &\\
		  &      &         & 0.4822 &        & 0.06667 &          & &\\
		2 & 1.10 & 1.04881 &        & -0.108 &         & -0.13335 & &\\
		  &      & 		   & 0.4714 &        & 0.04    &          & 0.8 &\\
		3 & 1.15 & 1.07238 &        & -0.102 &         & 0.06665  & & 10.66733 \\
		  &      & 		   & 0.4612 &        & 0.05333 &          & 4.0002 &\\
		4 & 1.20 & 1.09544 &        & -0.094 &         & 1.0667   & &\\
		  &      & 		   & 0.4518 &        & 0.26667 &          & &\\
		5 & 1.25 & 1.11803 &        & -0.09  &         &          & &\\
		  &      & 		   & 0.4428 &        &         &          & &\\
		6 & 1.30 & 1.14017 &        &        &         &          & &\\
		\hline
	\end{tabular}

	\textbf{Inciso b):}
	Para esta parte, cabe destacar que tranquilamente se puede usar una diferencia de orden 1 (y por ende un polinomio de grado 1) para calcular los valores (interpolación lineal).
	
	Es mucho más conveniente usar esto, antes que hacer muchos cálculos con el polinomio de grado 6.
	
	Entonces, usamos para $1.01$ los datos de 1.00 y 1.05
	
	\begin{tabular}{|c c c|}
		\hline
		$x_i$ & $y_i$ & O1\\
		\hline
		1.00 & 1.0000 & \\
		& & 0.494\\
		1.05 & 1.02470 & \\
		\hline
	\end{tabular}
	
	Armamos el polinomio progresivo: $p(x)=1+0.494(x-1) \rightarrow p(1.01)=1.00494$. El valor real es $1.0049875...$
	
	Para 1.28 podemos usar los datos de 1.25 y 1.30
	
	\begin{tabular}{|c c c|}
		\hline
		$x_i$ & $y_i$ & O1\\
		\hline
		1.25 & 1.11803 & \\
		& & 0.4428\\
		1.30 & 1.14017 & \\
		\hline
	\end{tabular}

	Armamos el progresivo: $p(x)=1.11803+0.4428(x-1.25) \rightarrow p(1.28)=1.131314$. El valor real es 1.13137085
	
	Como vemos, el error es bastante pequeño, y nos ahorramos hacer muchas cuentas.
	
	\textbf{Inciso d):}\\
	Si sabemos que $\sqrt{x}=1.05$, entonces tenemos que fijarnos por los valores de $y$. Mirando la tabla, vemos que ese valor está entre los $x$ 1.10 y 1.15
	
	Armemos polinomio para estos valores:
	
	\begin{tabular}{|c c c|}
		\hline
		$x_i$ & $y_i$ & O1\\
		\hline
		1.10 & 1.04881 & \\
		& & 0.4714\\
		1.15 & 1.07238 & \\
		\hline
	\end{tabular}

	$p(x)=1.04881+0.4714(x-1.1)$
	
	Como queremos averiguar $x$, lo igualamos a $1.05$:
	
	$1.05 = 1.04881 + 0.4714x - 0.51854 \rightarrow 0.51973 = 0.4714x \rightarrow x=1.102524395$
	
	Si comparamos este valor, con el real: $x=1.1025$ (pasando la raíz cuadrada como potencia), vemos que se aproxima bastante la solución.
	
	\textbf{Inciso e):}\\
	Para este hay que hacer $p(1.32)$, y acá ya no podemos usar interpolación lineal. Hay que armar el polinomio con todas las diferencias:
	
	$p(x)=1+0.494(x-1)-0.118(x-1)(x-1.05)+0.06667(x-1)(x-1.05)(x-1.1)-0.13335(x-1)(x-1.05)(x-1.1)(x-1.15)+0.8(x-1)(x-1.05)(x-1.1)(x-1.15)(x-1.2)+10.66733(x-1)(x-1.05)(x-1.1)(x-1.15)(x-1.2)(x-1.25)$
	
	$p(1.32)=1.14932092$
	
	Como comentario: no conviene usar valores que están fuera de nuestro intervalo de trabajo en interpolación para encontrar valores. En este caso, estamos dentro de un margen de tolerancia de un 10\% del extremo, pero para valores más alejados, los valores que arroja la función van a tener muy poco que ver con el valor original de la función.
	
	\section{Ejercicio 12 (Newton-Gregory + k)}
	Halle $k \in \mathbb{R}$ para que exista un polinomio de grado 2 que pase por todos los puntos de la siguiente tabla, y halle dicho polinomio en forma normalizada:
	
	\begin{center}
		\begin{tabular}{|c|c|c|c|c|c|}
		\hline
		$x_i$ & -1 & 0 & 2 & 3 & 5 \\
		\hline
		$f(x_i)$ & 10 & 3 & 1 & $k$ & 28 \\
		\hline
	\end{tabular}
	\end{center}

	ATENTOS acá, ejercicio TÍPICO de examen!!!
	
	Para resolverlo hay que usar la cantidad NECESARIA de puntos para el grado de polinomio que nos piden. Recordar que se necesitan $n+1$ puntos. Claramente, hay que dejar al punto que contiene la $k$ fuera de nuestros cálculos.
	
	Así que vamos a usar la tabla:
	
	\begin{tabular}{|c c c c|}
		\hline
		$x_i$ & $y_i$ & O1 & O2\\
		\hline
		-1 & 10 & & \\
		& & -7 & \\
		0  & 3  & &2 \\
		& & -1 & \\
		2  & 1 & & \\
		\hline
	\end{tabular}

	En este caso, vamos a usar el polinomio regresivo:
	
	$p(x)=1-(x-2)+2(x-2)x$
	
	Lo normalizamos: $p(x)=1-x+2+2x^2-4x=2x^2-5x+3$
	
	Primero hay que verificar que el punto que no tuvimos en cuenta para los cálculos verifique (ya que el polinomio interpolante DEBE pasar por todos los puntos).
	
	$p(5)=28 \rightarrow $ verifica
	
	Por último, para hallar el valor de $k$, basta con hacer $p(3)$, en este caso.
	
	$p(3)=k \rightarrow k=6$.
	
	\section{Ejercicio 13 (Newton-Gregory parcial)}
	Dada la siguiente tabla de datos:
	\begin{center}
		\begin{tabular}{|c|c|c|c|c|c|c|}
			\hline
			$x_i$ & 0 & 1 & 4 & 5 & 7 & 8\\
			\hline
			$f(x_i)$ & 4 & 3 & 0 & k & 123 & 220\\
			\hline
		\end{tabular}
	\end{center}

	\begin{enumerate}[label=\alph*)]
		\item Halle, si es posible, el valor de $k \in \mathbb{R}$ tal que por todos los puntos dados pase un polinomio de grado 3.
		\item Con el valor de $k$ hallado en a), indique cuántos polinomios de grado 4 pasan por los puntos dados, y cuántos de grado 10 pasan por ellos. Justifique.
	\end{enumerate}

	Otro ejercicio típico de parcial, en este se incluye un punto teórico
	
	\textbf{Inciso a):}
	
	Procedemos al igual que en el ejercicio anterior, tomamos una tabla reducida de puntos. En este caso necesitamos 4.
	
	\begin{tabular}{|c c c c c|}
		\hline
		$x_i$ & $y_i$ & O1 & O2 & O3\\
		\hline
		0 & 4 & & & \\
		& & -1 & & \\
		1 & 3  & &0 & \\
		& & -1 & & 1\\
		4 & 0 & & 7 &\\
		& & 41 & & \\
		7 & 123 & & &\\
		\hline
	\end{tabular}

	Usamos el polinomio progresivo en este caso:
	
	$p(x)=4-x+0x(x-1)+x(x-1)(x-4)=4-x+x(x-1)(x-4)$
	
	De nuevo, verificamos que pase por el punto que no consideramos: $p(8)=4-8+8(8-1)(8-4)=220 \rightarrow$ verifica
	
	Por último, hallamos $k$ haciendo $p(5)$.
	
	$p(5)=k \rightarrow k=19$
	
	\textbf{Inciso b):}
	
	Punto teórico, por estos puntos no pasa NINGÚN polinomio de grado 4, ya que contradice el teorema de existencia y unicidad del polinomio interpolante. "Dados n+1 puntos, existe un único polinomio de grado menor o igual a n que pase por dichos puntos".
	
	Como corolario del enunciado anterior, existen infinitos polinomios de grado 10 que pasen por los puntos.
	
	\section{Ejercicio 14 (Teórico)}
	Indique si las siguientes proposiciones son Verdaderas o Falsas justificando \textit{(solo hacemos c, e y f)}:
	
	\begin{itemize}
		\item Si se dispone de una tabla con 6 valores $x_i$ y sus imágenes $f_i$, entonces existe hasta la diferencia finita progresiva de orden 6 de $f_0$.
	\end{itemize}

	Falso, como vimos hasta ahora, cuando tenemos $n$ puntos, entonces el polinomio máximo es de orden $n-1$, por ende las diferencias llegan hasta el orden $n-1$. Es decir, se llegaría hasta la diferencia de orden 5.

	\begin{itemize}
		\item Dada una tabla de n pares de valores $(x_i,y_i)$, es posible encontrar más de un polinomio de grado n que interpolen todos los datos.
	\end{itemize}

	Verdadero, para n pares de valores, es posible hallar más de un polinomio de grado n. Si en cambio dijera n+1 pares de valores, podríamos hallar un solo polinomio de grado n o menor.
	
	\begin{itemize}
		\item Dado un conjunto de n puntos $(x_i,y_i)$, existe una única función interpolante tal que \\ $f(x_i)=y_i, \forall i:1\dots n$
	\end{itemize}
	Falso. Nos dicen FUNCIÓN interpolante, no polinomio interpolante. Por ejemplo, para los puntos $(0,0), (1,1), (2,2), (3,3)$	sabemos que su polinomio interpolante sería la función identidad. Sin embargo, también podemos interpolar con $f(x)=x+\sen(2\pi x)$.
	
		\section{Ejercicio 17 (Newton-Gregory + a y b)}
	Indique los valores de $a$ y $b$ tal que el polinomio interpolante de los 5 puntos sea de grado 4, pero que por los primeros 4 puntos pase un polinomio de grado 2.
	
	\begin{tabular}{|c|c|c|c|c|c|}
		\hline
		$x_i$ & 1 & 2 & 3& 4&5\\
		\hline
		$f(x_i)$ & 0 & 7 & 24 & $a$ & $b$\\
		\hline
	\end{tabular}

	$a=\dots\dots$ (¿Es único? SI / NO) \hspace{2cm} $b=\dots\dots$ (¿Es único? SI / NO)
	
	Para resolver este ejercicio, vamos a empezar por lo último: hallar el polinomio de grado 2 para los primeros 4 puntos.
	
	\begin{tabular}{|c c c c|}
		\hline
		$x_i$ & $y_i$ & O1 & O2\\
		\hline
		1 & 0 & & \\
		& & 7 & \\
		2  & 7  & &5 \\
		& & 17 & \\
		3  & 24 & & \\
		\hline
	\end{tabular}

	Usamos polinomio progresivo: $p(x)=0+7(x-1)+5(x-1)(x-2)$
	
	Entonces, el valor de $a$ debe pertenecer a este polinomio y debe ser este valor ÚNICO, ya que cualquier otro hará que por esos puntos NO pase un polinomio de grado 2. 
	
	$p(4)=a \rightarrow a=51$
	
	Luego, para el valor de $b$ nos piden que pase un polinomio de grado 4. Para que se cumpla esto, hay que tener en cuenta que por los primeros 4 puntos ya pasa un polinomio de grado 2.
	
	Para que pase un polinomio de grado 4 por todos los puntos es necesario que se rompa el patrón de los primeros 4 puntos.
	
	Para eso, evaluamos que valor debe tener $b$ para que por los puntos siga pasando un polinomio de grado 2:
	
	$p(5)=b \rightarrow b=88$. Si $b=88$ entonces por los puntos pasaría un polinomio de grado 2, entonces para cualquier valor de $b$ que NO sea 88 va a pasar un polinomio de grado 4 por todos los puntos.
	
	Esto es una regla general. Cuando tenemos puntos que están alineados a un polinomio en particular, y luego agregamos o cambiamos un punto que NO siga el patrón, el polinomio interpolante de los puntos se va al máximo grado posible, que es lo que ocurre en este caso.
	
	(hacer gráfico de la lineal)
	
	\section{Ejercicio 19 (Newton-Gregory + k)}
	El polinomio que interpola a los puntos $(0,3),(0.5,5),(1,7)$ y $(k, 5k)$ es:
	
	\begin{enumerate}[label=\alph*)]
		\item de grado 3, $\forall k \in \mathbb{R}$
		\item nunca es de grado 3
		\item $\exists k$ tal que sea de grado 1
		\item $\exists k$ tal que sea de grado 2
	\end{enumerate}

	Hacemos la tabla de diferencias para puntos sin $k$:
	
	\begin{tabular}{|c c c c|}
		\hline
		$x_i$ & $y_i$ & O1 & O2\\
		\hline
		0 & 3 & & \\
		& & 4 & \\
		0.5  & 5  & &0 \\
		& & 4 & \\
		1  & 7 & & \\
		\hline
	\end{tabular}

	La diferencia de orden 2 es nula, esto nos da la pauta de que tenemos un polinomio de grado 1.
	
	$p(x)=3+4x$
	
	Veamos si con este polinomio encontramos algún $k$ que verifique
	
	$p(k)=5k \rightarrow 5k=3+4k \rightarrow k=3$
	
	Encontramos un valor de $k$, siendo $k=3$. Por ende, la respuesta correcta es la c.
	
	Con esto, descartamos la a, porque sabemso que si $k=3$, tenemos polinomio de grado 1.\\
	La b la descartamos ya que si $k\ne3$ el polinomio va a ser de grado 3.\\
	La d sabemos que no es ya que la diferencia da 0, y los puntos están alineados a una recta.
\end{document}
 