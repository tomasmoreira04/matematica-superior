\documentclass[11pt]{article}

\usepackage[margin=1in, headheight=14.5pt]{geometry}
\usepackage{amsfonts, amsmath, amssymb}
\usepackage[none]{hyphenat}
\usepackage{fancyhdr}
\usepackage[spanish]{babel}
\usepackage[spanish, calc]{datetime2}
\usepackage{fmtcount}
\usepackage{graphicx}
\usepackage{float}
\usepackage[nottoc, notlot, notlof]{tocbibind}
\usepackage{tocloft}
\usepackage[utf8]{inputenc}
\usepackage{parskip}
\usepackage{xcolor}
\usepackage{cancel}
\usepackage{textcomp}
\usepackage{pgfplots}
\usepackage{tikz}
\usetikzlibrary{datavisualization}
\usetikzlibrary{datavisualization.formats.functions}
\pgfplotsset{compat=1.15}
\usepackage{mathrsfs}
\usetikzlibrary{arrows}

\parindent 0ex

\pgfplotsset{width=10cm,compat=1.9}

\def\imj{\mathrm{j}}
\def\sen{\mathrm{sen}}

\newcommand{\lapl}[1]{\mathscr{L} \lbrace {#1} \rbrace}
\newcommand{\ilapl}[1]{\mathscr{L}^{-1} \lbrace {#1} \rbrace}

\renewcommand\cftsecleader{\cftdotfill{\cftdotsep}}
\renewcommand{\baselinestretch}{1.1}
\newcommand*\circled[1]{\tikz[baseline=(char.base)]{
		\node[shape=circle,draw,inner sep=2pt] (char) {#1};}}
	
\newcommand{\highlight}[2]{\colorbox{#1}{$\displaystyle #2$}}

\graphicspath{{\ProjectRoot/commons/img/}}

\newcommand*{\ProjectRoot}{../../matematica-superior}


\begin{document}
		
	\begin{titlepage}
		\begin{center}
			\vspace*{0.5cm}
			\Large{\textbf{Universidad Tecnológica Nacional}}\\
			\Large{\textbf{Facultad Regional Buenos Aires}}\\
			\begin{center}
				\includegraphics[scale=0.4]{logoutn.png}
			\end{center}
			\vfill
			\line(1,0){400}\\
			\vspace*{0.3cm}
			\huge{\textbf{Matemática Superior}}\\
			\Large{\textbf{Unidad 3: Transformada de Laplace}}\\
			\large{Ejercicios resueltos}
			\line(1,0){400}\\
			\vfill
			Tomás Moreira \\
			
			\DTMnewdatestyle{mydate}{%
				\renewcommand{\DTMdisplaydate}[4]{%
					\DTMMonthname{##2} \number##1
				}
				\renewcommand{\DTMDisplaydate}{\DTMdisplaydate}
			}
			
			\DTMsetdatestyle{mydate}
			\today
				
				
		\end{center}
	\end{titlepage}

	\tableofcontents
	\thispagestyle{empty}
	\clearpage

	\setcounter{page}{1}
	\section{Introducción}
	\section{Transformada directa de Laplace - Propiedades}
	\subsection{Ejercicio 1d}
	Los ejercicios que pertenecen al 1 son todos para aplicar la propiedad de linealidad y luego transformar por la tabla. Vamos a realizar uno a modo de ejemplo, y de paso nos va a servir para establecer una pauta para después.
	
	d) $y(t)=4\cos(3t)-5\sen(2t)\quad;\quad t>0$
	
	Nos piden hallar la Transformada de Laplace.
	
	Recuerden que a la función temporal, por convención, la denotamos con letra minúscula y está en el dominio del tiempo, mientras que su transformada la escribimos con mayúscula, y está en el dominio de Laplace. La variable que se usa comúnmente para las Transformadas de Laplace es $s$.
	
	Entonces, arranquemos a resolver, aplicando transformada a ambos lados.
	
	$\displaystyle \lapl{y(t)}=\lapl{4\cos\left(3t\right) - 5\sen\left(2t\right)}$
	
	Del lado izquierdo, tenemos que, por definición, $y(t)$ pasa a ser $Y(s)$ y, del lado derecho, vamos a aplicar linealidad.
	
	$\displaystyle Y(s)=4\lapl{\cos\left(3t\right)}-5\lapl{\sen\left(2t\right)}$
	
	Luego, revisando la tabla de transformadas, podemos transformar de forma directa:
	
	\fcolorbox{black}{yellow}{$\displaystyle Y(s)=\frac{4s}{s^2+9}-\frac{10}{s^2+4}$}
	
	\textbf{ACLARACIÓN:} a partir de ahora la propiedad de linealidad la aplicaremos siempre sin mencionarla.
	
	\subsection{Ejercicios 2de}
	Utilizando las propiedades de la Transformada de Laplace, halle $\lapl{f(t)}$
	
	\subsubsection{2d}
	$f(t)=\left(t-1\right)^{4}\; ;\quad t>1$
	
	Los ejercicios del 2 implican todos aplicar alguna o algunas de las propiedades de la Transformada.
	
	En este caso, vemos que la función está desplazada una unidad hacia la derecha, y como consecuencia, y también el dominio está definido para los $t>1$. Por lo que podemos aplicar la \textbf{\underline{segunda propiedad}}\\
	\textbf{\underline{de la traslación}}.
	
	Recordemos que decía la propiedad:
	
	Si $ \displaystyle
	F(s)=\lapl{f(t)}\;\; \wedge\;\; 
	g(t) = 
	\begin{cases}
		f(t-a) &\quad\text{si}\; t > a\\
		0 &\quad\text{si}\; t < a \\
	\end{cases}
	\implies \lapl{g(t)}=e^{-as}F(s)$
	
	Por ende, aunque parezca un dato irrelevante, el $t>1$ termina siendo indispensable para aplicar esta propiedad. De este modo, si dijera $t>0$ como la gran mayoría de las funciones, \textbf{\underline{no} se podría aplicar la propiedad}.
	Considerando esto, la transformada se calcularía:
	
	Hacemos $a=1$
	
	Y calculamos la transformada de Laplace de la función $f$ ``sin considerar el desplazamiento".
	
	$\displaystyle F(s)=\frac{1}{s^2}$, luego aplicamos la propiedad de traslación, para que finalmente quede:
	
	\fcolorbox{black}{yellow}{$\displaystyle F(s)=\frac{24e^{-s}}{s^5}$}
	
	\subsubsection{2e}
	$f(t)=e^{-2t}\sen (2t)+t^{2}e^{3t} \;;\; t>0$
	
	Para este ejercicio hay que poner en práctica la primera propiedad de la traslación.
	
	Recordémosla:
	
	Sea $F(s)=\lapl{f(t)} \implies \lapl{e^{at}f(t)}=F(s-a)$
	
	Esto quiere decir que multiplicar por una exponencial en el dominio del tiempo, genera un desplazamiento en el dominio de Laplace.
	
    ``En criollo'' podemos decir que transformamos la función sin considerar la exponencial, y luego, reemplazamos $s$ por $s-a$.
    
    Para este caso entonces, primero hallamos $\lapl{\sen(2t)}$ y $\lapl{t^{2}}$. Estas dos, salen por tabla:
    
    $\displaystyle \lapl{\sen(2t)}=\frac{2}{s^{2}+4}$
    
    $\displaystyle \lapl{t^{2}}=\frac{2}{s^{3}}$
    
    Y, como dijimos, ahora reemplazamos $s$ por $s-a$. En el caso de la sinusoide, $a=-2$ y para el segundo término $a=3$.
    
    Nos queda finalmente:
    
    \fcolorbox{black}{yellow}{$\displaystyle F(s)=\frac{2}{(s+2)^{2}+4}+\frac{2}{(s-2)^{3}}$}
    
    \subsection{Ejercicios 3abcdef}
    Calcule las transformadas de Laplace aplicando adecuadamente alguna propiedad conveniente o algún paso algebraico:
    
    \subsubsection{3a}
    $\displaystyle y(t)=\frac{1}{a^{2}} \;;\;0<t<a$
    
    Este ejercicio tiene la particularidad de que no está en el dominio de la integral de Laplace (desde 0 a infinito), sino que está acotada entre a un cierto dominio del tiempo. Para poder resolverlo, debemos recurrir a la función escalón unitario, o función de Heaviside.
    
    Un posible gráfico de la función que se nos presenta en el ejercicio, es:
    
    \includegraphics[scale=0.6]{03-TransformadaDeLaplace/3a1.png}
    
    Mirando esto, podemos sacar partido de la función escalón unitario, y redefinir a la función en términos de ella:
    
    $\displaystyle y(t)=\frac{1}{a^{2}}E(t)-\frac{1}{a^{2}}E(t-a)$
    
    Básicamente, lo que estamos haciendo es multiplicar a nuestra función por una función escalón unitario, lo cual nos daría todo el trozo que corresponde a su dominio, y luego, le restamos la parte que no consideramos (ya que para valores mayores a $a$, la función es 0).
    
    Algo a considerar es que no hace falta ``restar'' la parte de $t<0$, ya que la función escalón unitario está definida para $t \geq 0$.
    
    Luego, transformamos la función que tenemos. 
    
    \textbf{\underline{NOTA:}} 
    
    $\displaystyle \lapl{E(t)}=\frac{1}{s}$
    
    $\displaystyle \lapl{E(t-a)}=\frac{e^{-as}}{s}$
    
    $\displaystyle \lapl{y(t)}=\frac{1}{a^{2}} \cdot \frac{1}{s}+ \frac{1}{a^{2}} \cdot \frac{e^{-as}}{s} \implies$
    \fcolorbox{black}{yellow}{$\displaystyle Y(s)=\frac{1-e^{-as}}{a^{2}s}$}
    
    \subsubsection{3b}
    $y(t)=\sen \left(5t+\frac{\pi}{3} \right)\;;\;t>0$
    
    Para resolver este ejercicio, hay que darse cuenta que no se puede aplicar la segunda propiedad de la traslación, ya que no se cumplen las condiciones para poder aplicarla.
    
    Resulta, entonces, que hay que hacer un paso extra para poder transformar esta función. En este caso, basta con aplicar el seno de la suma:
    
    $\displaystyle y(t)=\sen \left(5t\right)\cdot \cos\left(\frac{\pi}{3}\right)+ \sen\left(\frac{\pi}{3}\right)\cdot\cos\left(5t\right)$
    
    Calculamos los números: $\displaystyle y(t)=\frac{1}{2}\sen(5t)+\frac{\sqrt{3}}{2}cos(5t)$
    
    Transformamos usando la tabla: $\displaystyle Y(s)=\frac{1}{2}\frac{5}{s^{2}+25}+\frac{\sqrt{3}}{2}\frac{s}{s^{2}+25}\implies$ \fcolorbox{black}{yellow}{$\displaystyle Y(s)=\frac{\sqrt{3}s+5}{2\left(s^{2}+25\right)}$}
    
    \subsubsection{3c}
    $y(t)=\cos^{3}(t)$
    
    Se puede proceder de muchas maneras para resolver este ejercicio. Yo voy a optar por transformar esta expresión en una equivalente.
    
    Si usamos la forma exponencial del coseno: (recordar) $\displaystyle cos(a)=\frac{e^{aj}+e^{-aj}}{2}$
    
    $\displaystyle y(t)=\left(\frac{e^{jt}+e^{-jt}}{2}\right)^{3}
    =\frac{e^{3jt}}{8}+\frac{3e^{2jt}\cdot e^{-jt}}{8}+\frac{3e^{jt}\cdot e^{-2jt}}{8}+\frac{e^{-3jt}}{8}$
    
    Operando, ordenando y agrupando convenientemente: $\displaystyle y(t)=
    \frac{\highlight{yellow}{ e^{3jt}+e^{-3jt}}}{\highlight{yellow}{2}\cdot4}+\frac{3\left( \highlight{green}{e^{jt}+e^{-jt}} \right)} {\highlight{green}{2}\cdot4}$
    
    Si nos damos cuenta, lo sombreado con amarillo y con verde, representan también la forma exponencial de un coseno, entonces, lo reemplazamos:
    
    $\displaystyle y(t)=\frac{1}{4}\cos(3t)+\frac{3}{4}\cos(t)$
    
    Ahora, estamos en condiciones de transformar directamente: \fcolorbox{black}{yellow}{$\displaystyle Y(s)=\frac{1}{4}\cdot \frac{s}{s^{2}+9}+\frac{3}{4}\cdot \frac{s}{s^{2}+1}$}
    
    \subsubsection{3d}
    $y(t)=a^{t} \; ; \; t>0 \; ; \; a \in \mathbb{R}^{+}$
    
    Aplicamos logaritmo a ambos miembros
    
    $\ln(y(t))=t \ln(a)$
    
    $ \displaystyle e^{\ln(y(t))}=e^{t\cdot ln(a)} \implies y(t)=e^{\ln(a)t}$
    
    ¡Que no panda el cúnico! el $\ln(a)$ es un número! Entonces, podemos transformar directamente usando la tabla para la exponencial:
    
    \fcolorbox{black}{yellow}{$\displaystyle Y(s)=\frac{1}{s-\ln(a)}$}
    
    \subsubsection{3e}
    $y(t)=(t-1)^{4} \; ; \; t>0$
    
    Este ejercicio es una variante del ejercicio 2d. ¿Cuál es la diferencia? El dominio donde está definida la función. Para el ejercicio 2d, decía claramente $t>1$, de forma tal que podíamos aplicar la segunda propiedad de la traslación.
    
    En este caso, no coincide el desplazamiento de la función, con el dominio. Por ende, no se cumplen las hipótesis para aplicar la propiedad.
    
    ¿Qué hacemos? Desarrollamos el binomio y transformamos el resultado. Entonces:
    
    $y(t)=t^{4}-4t^{3}+6t^{2}-4t+1$
    
    $\displaystyle \lapl{y(t)}=\frac{4!}{s^{5}}-4\frac{3!}{s^{4}}+6\frac{2!}{s^{3}}-4\frac{1!}{s^{2}}+\frac{1}{s}$
    
    \fcolorbox{black}{yellow}{$\displaystyle Y(s)=\frac{24}{s^{5}}-\frac{24}{s^{4}}+\frac{12}{s^{3}}-\frac{4}{s^{2}}+\frac{1}{s}$}
	
	\subsubsection{3f}
	$y(t)=\sen(5t)\cos(2t)\;;\;t>0$
	
	Para este último ejercicio no tenemos una transformada directa. Entonces,toca aplicar alguna identidad trigonométrica, ya que tenemos un seno y un coseno.
	
	Tengamos en cuenta: \fcolorbox{black}{white}{$\displaystyle \sen(a)\cdot \cos(b) = \frac{\sen(a-b)+\sen(a+b)}{2}$}
	
	Usamos la identidad:
	$\displaystyle y(t)=\frac{1}{2}\sen(5t-2t)+\frac{1}{2} \sen(5t+2t)$
	
	$\displaystyle y(t)=\frac{1}{2}\sen(3t)+\frac{1}{2} \sen(7t)$
	
	Estas dos funciones ya son transformables:
	
	$\displaystyle Y(s)=\frac{3}{2(s^{2}+9)}+\frac{7}{2(s^{2}+49)}=\frac{3(s^{2}+49)+7(s^{2}+9)}{2(s^{2}+9)(s^{2}+49)}=\frac{3s^2+147+7s^2+63}{2(s^2+9)(s^2+49)}$
	
	Por último: \fcolorbox{black}{yellow}{$\displaystyle Y(s)=\frac{5s^2+105}{(s^2+49)(s^2+9)}$}
	
	\section{Antitransformada de Laplace - Propiedades}
	\section{Resolución de ecuaciones diferenciales}
	\section{Evaluación de integrales}
	\section{Ejercicios bonus}
	\section{Ejercicios para curiosos}
\end{document}
