\documentclass[11pt]{article}

\usepackage[margin=1in, headheight=14.5pt]{geometry}
\usepackage{amsfonts, amsmath, amssymb}
\usepackage[none]{hyphenat}
\usepackage{fancyhdr}
\usepackage[spanish]{babel}
\usepackage[spanish, calc]{datetime2}
\usepackage{fmtcount}
\usepackage{graphicx}
\usepackage{float}
\usepackage[nottoc, notlot, notlof]{tocbibind}
\usepackage{tocloft}
\usepackage[utf8]{inputenc}
\usepackage{parskip}
\usepackage{xcolor}
\usepackage{cancel}
\usepackage{pgfplots}
\usepackage{tikz}
\usetikzlibrary{datavisualization}
\usetikzlibrary{datavisualization.formats.functions}

\parindent 0ex

\pgfplotsset{width=10cm,compat=1.9}

\def\imj{\mathrm{j}}

\renewcommand\cftsecleader{\cftdotfill{\cftdotsep}}
\renewcommand{\baselinestretch}{1.1}
\newcommand*\circled[1]{\tikz[baseline=(char.base)]{
		\node[shape=circle,draw,inner sep=2pt] (char) {#1};}}

\graphicspath{{C:/Users/tomas/OneDrive/Escritorio/LATEX/matematica-superior/commons/img/}}


\begin{document}
		
	\begin{titlepage}
		\begin{center}
			\vspace*{0.5cm}
			\Large{\textbf{Universidad Tecnológica Nacional}}\\
			\Large{\textbf{Facultad Regional Buenos Aires}}\\
			\begin{center}
				\includegraphics[scale=0.4]{logoutn.png}
			\end{center}
			\vfill
			\line(1,0){400}\\
			\vspace*{0.3cm}
			\huge{\textbf{Matemática Superior}}\\
			\Large{\textbf{Unidad 1: Números complejos}}\\
			\large{Ejercicios resueltos}
			\line(1,0){400}\\
			\vfill
			Tomás Moreira \\
			
			\DTMnewdatestyle{mydate}{%
				\renewcommand{\DTMdisplaydate}[4]{%
					\DTMMonthname{##2} \number##1
				}
				\renewcommand{\DTMDisplaydate}{\DTMdisplaydate}
			}
			
			\DTMsetdatestyle{mydate}
			\today
				
				
		\end{center}
	\end{titlepage}

	\tableofcontents
	\thispagestyle{empty}
	\clearpage

	\setcounter{page}{1}
	\section{Introducción}
	¡Hola! Bienvenido a esta guía que busca mostrar la resolución de algunos ejercicios de la primera unidad de la materia: Números Complejos.
	El documento está hecho al 100\% con \AmS-\LaTeX. Esperemos que te sea de utilidad, y no dudes en consultar cualquier duda.\\
	Como repaso de álgebra, vemos las operaciones más importantes y tratamos con estos números en sus distintas formas. Estos nos van a servir
	en futuras unidades, como Serie de Fourier y Función de transferencia.
	\section{Forma binómica}
	\subsection{Ejercicio 1d}
		Resuelva la siguiente operación en forma binómica:
		
		\begin{center}
			$\displaystyle{\mathrm{Im}\left[\frac{\left(4+7\imj\right)\cdot\left(6-2\imj\right)}{2\imj}\right]+4\imj=}$
		\end{center}
		
		Para resolver este ejercicio necesitamos resolver primero todo lo que se encuentre dentro del operador '$\mathrm{Im}$'.
		Entonces, realizamos primero la multiplicación en el numerador:
		$$\left(4+7\imj\right)\cdot\left(6-2\imj\right)=$$
		$$4\cdot6+4\cdot\left(-2\imj\right)+7\imj\cdot6+7\imj\cdot\left(-2\imj\right)=$$
		$$24-8\imj+42\imj-14\imj^{2}$$
		\begin{center}
			Recordemos que $\imj^{2}=-1$, entonces aplicando esta igualdad, podemos reemplazar donde corresponde:
		\end{center}
		$$24-8\imj+42\imj-14\cdot\left(-1\right)=$$
		$$24-8\imj+42\imj+14=$$
		$$\boxed{38-34\imj}$$
		\begin{center}
			Esta es la expresión resultante, por lo que la reemplazamos en la original:
		\end{center}
		$$\displaystyle{\mathrm{Im}\left[\frac{38-34\imj}{2\imj}\right]+4\imj=}$$
		\begin{center}
			El paso siguiente es realizar la división de los números complejos, para eso multiplicamos y dividimos por $\imj$:
		\end{center}
		$$\frac{38-34\imj}{\imj}=$$
		$$\frac{\left(38-34\imj\right)\cdot\imj}{2\imj\cdot\imj}=$$
		$$\frac{38\imj+34}{-2}=\boxed{-17-19\imj}$$
		\begin{center}
			Volvemos a reemplazar lo obtenido en la expresión original:
		\end{center}
		$$\mathrm{Im}\left[-17-19\imj\right]+4j=\cdots$$
		Ahora, si tenemos a un número complejo de la forma $z=a+b\imj$, definimos $\mathrm{Im}(z)=b$, \textbf{\underline{SIN}} la $\imj$.
		
		Entonces, aplicando esto a la expresión obtenemos el resultado final:
		\begin{center}
			\fcolorbox{black}{yellow}{$-19+4\imj$}
		\end{center}
	\subsection{Ejercicio 5a}
		Determine el conjunto de los complejos que cumplan las siguientes condiciones:
		
		\begin{center}
			{\large \textbf{Que su cuadrado sea igual a su conjugado}}
		\end{center}
				
		Para resolver este ejercicio vamos a traducir lo que dice el enunciado en notación literal.
		
		Consideremos un número complejo $z=x+y\imj$
		
		Lo que nos piden, entonces es:
		$$z^{2}=\overline{z}$$
		Recordemos la definición del conjugado de un número complejo:
		\begin{center}
			Sea $z=x+y\imj$, entonces $\overline{z}=x-y\imj$
		\end{center}
		
		Es decir, dado un número complejo, su conjugado es el resultado de cambiarle el signo a su parte imaginaria.
		
		Con esto dicho, empecemos a operar:
		$$\left(x+y\imj\right)^{2}=x-y\imj$$
		$$x^{2}+2xy\imj+y^{2}\imj^{2}=x-y\imj$$
		\begin{center}
			Recordemos que $\imj^{2}=-1$ y acomodemos un poco la ecuación:
		\end{center}
		$$x^{2}-y^{2}+2xy\imj=x-y\imj$$
		Ahora bien, para que esta igualdad se cumpla, hay que recordar la igualdad de números complejos:
		\begin{center}
			Sean $r=a+b\imj \wedge s=c+d\imj$ entonces, estos dos números complejos son iguales sí y solo sí $a=c \wedge b=d$ 
		\end{center}
		\begin{center}
			Aplicamos lo previamente mencionado y vamos a obtener dos ecuaciones:
		\end{center}
		$$\left\{\begin{matrix}
		x^2-y^2=x \hspace{0.5cm} \circled{1} \\ 2xy=-y \hspace{0.8cm} \circled{2}
		\end{matrix}\right.$$
		\begin{center}
			Resolvamos primero para \circled{2}
			$$2xy=-y$$
			$$2x\cancel{y}=-\cancel{y}$$
			¡¡Un momento!! ¿Está bien simplificar esas $y$? \\
			Claro que sí, pero es importante tener la consideración de que $y$ NO puede ser cero.\\
			Por lo que nuestro ejercicio se va a dividir en dos. Una parte considerando que $y$ sea cero y la otra, en la que vamos a considerar que \textbf{no} sea cero. \\
			\vspace{0.35cm}
			\underline{\textbf{Si $y\neq0$:}}
			$$2x\cancel{y}=-\cancel{y}$$
			$$2x=-1$$
			$$\boxed{x=-\frac{1}{2}}$$
			Reemplacemos este resultado en \circled{1}
			$${\left(-\frac{1}{2}\right)}^{2}-y^{2}=-\frac{1}{2}$$
			$$\frac{1}{4}-y^{2}=-\frac{1}{2}$$
			$$y^{2}=\frac{3}{4}$$
			$$\boxed{y=\pm\frac{\sqrt{3}}{2}}$$
			Por lo que finalmente para este camino, tenemos dos soluciones:
			$$\boxed{\left( x=-\frac{1}{2} \wedge y=\frac{\sqrt{3}}{2}\right)  \vee \left( x=-\frac{1}{2}\wedge y=-\frac{\sqrt{3}}{2}\right)}$$
			
			Ahora consideremos el otro camino. \\
			\textbf{\underline{Si $y=0$:}} \\
			En \circled{1}:
			$$x^2-0^2=x$$
			$$x^2-x=0$$
			$$x(x-1)=0$$
			Por lo tanto:
			$$\boxed{x=0 \vee x=1}$$
			
			Recapitulando, en total obtenemos cuatro soluciones, que escritas en notación de par ordenado son:
			\begin{center}
				\fcolorbox{black}{yellow}{$\displaystyle{
						z=\left(-\frac{1}{2},\frac{\sqrt{3}}{2}\right) \vee
						z=\left(-\frac{1}{2},-\frac{\sqrt{3}}{2}\right) \vee
						z=\left(1,0\right) \vee
						z=\left(0,0\right)
					}$}
			\end{center}
		\end{center}		
	\subsection{Ejercicios 7cdh}
	Describa y construya la gráfica del lugar geométrico representado por cada una de las siguientes ecuaciones: (considere $z=x+y\imj$)
	\subsubsection{7c}
	$\mathrm{c)}\;z\left(\overline{z}+2\right)=3$
	
	Empecemos aplicando propiedad distributiva:
	$$z\overline{z}+2z=3$$
	\begin{center}
		Teniendo en cuenta que $z\overline{z}=\left\vert z \right\vert ^{2}=x^2+y^2$
	\end{center}
	$$x^2+y^2+2(x+y\imj)=3$$
	$$x^2+2x+y^2+2y\imj=3$$
	\begin{center}
		Nuevamente debemos usar la igualdad entre complejos para resolver, entonces obtenemos dos ecuaciones:
		$$\left\lbrace\begin{matrix}
		x^2+2x+y^2=3 \\
		2y=0
		\end{matrix}\right.$$
		$$\left\lbrace\begin{matrix}
		x^2+2x+y^2=3 \\
		y=0
		\end{matrix}\right.$$
		Como $y=0$, reemplazamos en la otra ecuación:
		$$\left\lbrace\begin{matrix}
		x^2+2x-3=0 \\
		y=0
		\end{matrix}\right.$$
		Resolviendo la ecuación cuadrática obtenemos:
		$$x=1\vee x=-3$$
	\end{center}
	Por lo que nuestras soluciones son:
	\begin{center}
		\fcolorbox{black}{yellow}{$z=(1,0) \vee z=(-3,0)$}
	\end{center}
	La gráfica sería, entonces: \\
	\begin{center}
		\begin{tikzpicture}
			\begin{axis}[
			unit vector ratio*=1 1 1,
			axis lines=middle,
			xlabel={$x$}, ylabel={$y$},
			ytick={-1,0,1},
			xmin=-5, xmax=2,
			ymin=-2, ymax=2,
			]
			\addplot [only marks] table {
				-3  0
				1   0
			};
			\end{axis}
		\end{tikzpicture} 
	\end{center}
	\section{Forma polar}
	\section{Raíces n-ésimas}
	\section{Logaritmo natural y exponenciales complejas}
	\section{Ejercicios combinados}
	\section{Superposición de señales senoidales de igual frecuencia}
\end{document}
