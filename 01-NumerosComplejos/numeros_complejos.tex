\documentclass[11pt]{article}

\usepackage[margin=1in, headheight=14.5pt]{geometry}
\usepackage{amsfonts, amsmath, amssymb}
\usepackage[none]{hyphenat}
\usepackage{fancyhdr}
\usepackage[spanish]{babel}
\usepackage[spanish, calc]{datetime2}
\usepackage{fmtcount}
\usepackage{graphicx}
\usepackage{float}
\usepackage[nottoc, notlot, notlof]{tocbibind}
\usepackage{tocloft}
\usepackage[utf8]{inputenc}
\usepackage{parskip}
\usepackage{xcolor}
\usepackage{cancel}
\usepackage{pgfplots}

\parindent 0ex

\pgfplotsset{width=10cm,compat=1.9}

\def\imj{\mathrm{j}}

\renewcommand\cftsecleader{\cftdotfill{\cftdotsep}}
\renewcommand{\baselinestretch}{1.1}

\graphicspath{{C:/Users/tomas/OneDrive/Escritorio/LATEX/matematica-superior/commons/img/}}


\begin{document}
		
	\begin{titlepage}
		\begin{center}
			\vspace*{0.5cm}
			\Large{\textbf{Universidad Tecnológica Nacional}}\\
			\Large{\textbf{Facultad Regional Buenos Aires}}\\
			\begin{center}
				\includegraphics[scale=0.4]{logoutn.png}
			\end{center}
			\vfill
			\line(1,0){400}\\
			\vspace*{0.3cm}
			\huge{\textbf{Matemática Superior}}\\
			\Large{\textbf{Unidad 1: Números complejos}}\\
			\large{Ejercicios resueltos}
			\line(1,0){400}\\
			\vfill
			Tomás Moreira \\
			
			\DTMnewdatestyle{mydate}{%
				\renewcommand{\DTMdisplaydate}[4]{%
					\DTMMonthname{##2} \number##1
				}
				\renewcommand{\DTMDisplaydate}{\DTMdisplaydate}
			}
			
			\DTMsetdatestyle{mydate}
			\today
				
				
		\end{center}
	\end{titlepage}

	\tableofcontents
	\thispagestyle{empty}
	\clearpage

	\setcounter{page}{1}
	\section{Introducción}
	¡Hola! Bienvenido a esta guía que busca mostrar la resolución de algunos ejercicios de la primera unidad de la materia: Números Complejos.
	El documento está hecho al 100\% con \AmS-\LaTeX. Esperemos que te sea de utilidad, y no dudes en consultar cualquier duda.\\
	Como repaso de álgebra, vemos las operaciones más importantes y tratamos con estos números en sus distintas formas. Estos nos van a servir
	en futuras unidades, como Serie de Fourier y Función de transferencia.
	\section{Forma binómica}
	\subsection{Ejercicio 1d}
		Resuelva la siguiente operación en forma binómica:
		
		\begin{center}
			$\displaystyle{\mathrm{Im}\left[\frac{\left(4+7\imj\right)\cdot\left(6-2\imj\right)}{2\imj}\right]+4\imj=}$
		\end{center}
		
		Para resolver este ejercicio necesitamos resolver primero todo lo que se encuentre dentro del operador '$\mathrm{Im}$'.
		Entonces, realizamos primero la multiplicación en el numerador:
		$$\left(4+7\imj\right)\cdot\left(6-2\imj\right)=$$
		$$4\cdot6+4\cdot\left(-2\imj\right)+7\imj\cdot6+7\imj\cdot\left(-2\imj\right)=$$
		$$24-8\imj+42\imj-14\imj^{2}$$
		\begin{center}
			Recordemos que $\imj^{2}=-1$, entonces aplicando esta igualdad, podemos reemplazar donde corresponde:
		\end{center}
		$$24-8\imj+42\imj-14\cdot\left(-1\right)=$$
		$$24-8\imj+42\imj+14=$$
		$$\boxed{38-34\imj}$$
		\begin{center}
			Esta es la expresión resultante, por lo que la reemplazamos en la original:
		\end{center}
		$$\displaystyle{\mathrm{Im}\left[\frac{38-34\imj}{2\imj}\right]+4\imj=}$$
		\begin{center}
			El paso siguiente es realizar la división de los números complejos, para eso multiplicamos y dividimos por $\imj$:
		\end{center}
		$$\frac{38-34\imj}{\imj}=$$
		$$\frac{\left(38-34\imj\right)\cdot\imj}{2\imj\cdot\imj}=$$
		$$\frac{38\imj+34}{-2}=\boxed{-17-19\imj}$$
		\begin{center}
			Volvemos a reemplazar lo obtenido en la expresión original:
		\end{center}
		$$\mathrm{Im}\left[-17-19\imj\right]+4j=\cdots$$
		Ahora, si tenemos a un número complejo de la forma $z=a+b\imj$, definimos $\mathrm{Im}(z)=b$, \textbf{\underline{SIN}} la $\imj$.
		
		Entonces, aplicando esto a la expresión obtenemos el resultado final:
		\begin{center}
			\fcolorbox{black}{yellow}{$-19+4\imj$}
		\end{center}
	\subsection{Ejercicio 5a}
		Determine el conjunto de los complejos que cumplan las siguientes condiciones:
		
		\begin{center}
			{\large \textbf{Que su cuadrado sea igual a su conjugado}}
		\end{center}
				
		Para resolver este ejercicio vamos a traducir lo que dice el enunciado en notación literal.
		
		Consideremos un número complejo $z=x+y\imj$
		
		Lo que nos piden, entonces es:
		$$z^{2}=\overline{z}$$
		Recordemos la definición del conjugado de un número complejo:
		\begin{center}
			Sea $z=x+y\imj$, entonces $\overline{z}=x-y\imj$
		\end{center}
		
		Es decir, dado un número complejo, su conjugado es el resultado de cambiarle el signo a su parte imaginaria.
		
		Con esto dicho, empecemos a operar:
		$$\left(x+y\imj\right)^{2}=x-y\imj$$
		$$x^{2}+2xy\imj+y^{2}\imj^{2}=x-y\imj$$
		\begin{center}
			Recordemos que $\imj^{2}=-1$ y acomodemos un poco la ecuación:
		\end{center}
		$$x^{2}-y^{2}+2xy\imj=x-y\imj$$
		Ahora bien, para que esta igualdad se cumpla, hay que recordar la igualdad de números complejos:
		\begin{center}
			Sean $r=a+b\imj \wedge s=c+d\imj$ entonces, estos dos números complejos son iguales sí y solo sí $a=c \wedge b=d$ 
		\end{center}
		\begin{center}
			Aplicamos lo previamente mencionado y vamos a obtener dos ecuaciones:
		\end{center}
		$$\left\{\begin{matrix}
		x^2-y^2=x \\ 2xy=-y
		\end{matrix}\right.$$
		\begin{center}
			\begin{tabular}{c | c}
				Si consideramos que $y\neq0$ & Si consideramos que $y=0$\\\hline
				$\left\{\begin{matrix}
				x^2-y^2=x \\ 2x\cancel{y}=-\cancel{y}
				\end{matrix}\right.$ &
				$\left\{\begin{matrix}
				x^2-y^2=x \\ y=0
				\end{matrix}\right.$ \\[0.5cm]
				$\left\{\begin{matrix}
				x^2-y^2=x \\ 2x=-1
				\end{matrix}\right.$ &
				$\left\{\begin{matrix}
				x^2-0=x \\ y=0
				\end{matrix}\right.$ \\[0.5cm]
				$\displaystyle{\left\{\begin{matrix}
					x^2-y^2=x \\ x=-\frac{1}{2}
					\end{matrix}\right.}$ &
				$\left\{\begin{matrix}
				x^2-x=0 \\ y=0
				\end{matrix}\right.$ \\[0.5cm]
				$\left\{\begin{matrix}
				\left(-\frac{1}{2}\right)^2-y^2=-\frac{1}{2} \\ x=-\frac{1}{2}
				\end{matrix}\right.$ &
				$\left\{\begin{matrix}
				x(x-1)=0 \\ y=0
				\end{matrix}\right.$ \\ [0.5cm]
				$\displaystyle{\left\lbrace\begin{matrix}
					y=\pm\frac{\sqrt{3}}{2} \\ x=-\frac{1}{2}
					\end{matrix}\right.}$ &
				$\left\{\begin{matrix}
				x=0 \vee x=1 \\ y=0
				\end{matrix}\right.$
			\end{tabular}
		\end{center}
		\pagebreak
		Finalmente, las soluciones son:
		\begin{center}
			\fcolorbox{black}{yellow}{$\displaystyle{
					z=\left(-\frac{1}{2},\frac{\sqrt{3}}{2}\right) \vee
					z=\left(-\frac{1}{2},-\frac{\sqrt{3}}{2}\right) \vee
					z=\left(1,0\right) \vee
					z=\left(0,0\right)
				}$}
		\end{center}
	\section{Forma polar}
	\section{Raíces n-ésimas}
	\section{Logaritmo natural y exponenciales complejas}
	\section{Ejercicios combinados}
	\section{Superposición de señales senoidales de igual frecuencia}
\end{document}
